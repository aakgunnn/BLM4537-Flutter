\documentclass[12pt,a4paper]{article}
\usepackage[utf8]{inputenc}
\usepackage{geometry}
\usepackage{graphicx}
\usepackage{hyperref}
\usepackage{listings}
\usepackage{xcolor}
\usepackage{booktabs}
\usepackage{float}
\usepackage{caption}
\usepackage{longtable}
\usepackage{fancyhdr}
\usepackage{tocloft}

\geometry{margin=2.5cm}
\hypersetup{colorlinks=true, linkcolor=blue, urlcolor=blue}

% Kod stili
\lstset{
    basicstyle=\ttfamily\small,
    breaklines=true,
    frame=single,
    backgroundcolor=\color{gray!10},
    keywordstyle=\color{blue},
    commentstyle=\color{green!60!black},
    stringstyle=\color{orange},
    showstringspaces=false
}

\pagestyle{fancy}
\fancyhf{}
\rhead{Kutuphane Yonetim Sistemi - Mobil}
\lhead{BLM4537 Final Projesi}
\rfoot{Sayfa \thepage}

\begin{document}

% ==================== KAPAK SAYFASI ====================
\begin{titlepage}
    \centering
    \vspace*{1cm}
    
    {\Large\textbf{ANKARA UNIVERSITESI}}\\[0.3cm]
    {\large Muhendislik Fakultesi}\\[0.3cm]
    {\large Bilgisayar Muhendisligi Bolumu}\\[2cm]
    
    {\Huge\textbf{KUTUPHANE YONETIM SISTEMI}}\\[0.5cm]
    {\Large Flutter Mobil Uygulama}\\[2cm]
    
    {\large\textbf{BLM4537 - Mobil Programlama Final Projesi}}\\[2cm]
    
    \begin{tabular}{ll}
        \textbf{Ogrenci:} & Ahmet Akgun \\
        \textbf{Ogrenci No:} & 22290602 \\
        \textbf{Danisman:} & [HOCANIN ADI] \\
    \end{tabular}\\[2cm]
    
    \vfill
    
    % VIDEO LINKI
    \fbox{\parbox{0.8\textwidth}{
        \centering
        \textbf{PROJE TANITIM VIDEOSU}\\[0.3cm]
        \url{https://BURAYA-VIDEO-LINKINI-KOYUN}\\[0.2cm]
        \small (Video izinli ve herkese acik olmalidir)
    }}
    
    \vspace{1cm}
    {\large Ocak 2026}
\end{titlepage}

% ==================== ICINDEKILER ====================
\tableofcontents
\newpage

% ==================== OZET ====================
\section{Ozet}
Bu proje, Flutter framework kullanilarak gelistirilen cross-platform bir kutuphane yonetim sistemi mobil uygulamasidir. Uygulama, .NET Core Web API backend ile iletisim kurarak kitap yonetimi, odunc alma/iade islemleri ve istatistik goruntuleme gibi islevleri saglamaktadir.

\textbf{Kullanilan Teknolojiler:}
\begin{itemize}
    \item \textbf{Framework:} Flutter 3.x, Dart
    \item \textbf{State Management:} Riverpod
    \item \textbf{HTTP Client:} Dio
    \item \textbf{Navigation:} go\_router
    \item \textbf{Grafikler:} fl\_chart
    \item \textbf{Code Generation:} Freezed, json\_serializable
    \item \textbf{Secure Storage:} flutter\_secure\_storage
\end{itemize}

% ==================== GIRIS ====================
\section{Giris ve Amac}

\subsection{Projenin Amaci}
Bu mobil uygulamanin temel amaci, kutuphane yonetim sistemine mobil cihazlardan erisim saglamaktir. Uygulama su ozellikleri sunmaktadir:

\begin{enumerate}
    \item Kitap kataloguna goz atma ve arama
    \item Kullanici kayit ve giris islemleri
    \item Kitap odunc talebi olusturma
    \item Odunc gecmisini goruntuleme
    \item Admin paneli ile yonetim islemleri
    \item Istatistik ve raporlama (grafikler)
\end{enumerate}

\subsection{Kapsam}
Uygulama iki kullanici rolunu desteklemektedir:
\begin{itemize}
    \item \textbf{Uye (Member):} Kitap goruntuleme, odunc talep etme, profil yonetimi
    \item \textbf{Admin:} Tum islemler + kitap/odunc yonetimi, istatistik goruntuleme
\end{itemize}

% ==================== MIMARI ====================
\section{Uygulama Mimarisi}

\subsection{Katmanli Yapi}
Uygulama, Clean Architecture prensiplerine uygun katmanli bir yapida tasarlanmistir:

\begin{figure}[H]
\centering
\begin{tabular}{|c|}
\hline
\textbf{Screens (UI Layer)} \\
Auth, User, Admin ekranlari \\
\hline
\textbf{Widgets (Reusable Components)} \\
BookCard, LoadingWidget, vs. \\
\hline
\textbf{Providers (State Management)} \\
AuthProvider, BookProvider, LoanProvider, StatisticsProvider \\
\hline
\textbf{Services (Business Logic)} \\
ApiClient, AuthService, BookService, LoanService \\
\hline
\textbf{Models (Data Layer)} \\
BookModel, UserModel, LoanModel, StatisticsModel \\
\hline
\end{tabular}
\caption{Flutter Uygulama Mimarisi}
\end{figure}

\subsection{Proje Dosya Yapisi}
\begin{lstlisting}[language=bash]
lib/
|-- main.dart                 # Uygulama giris noktasi
|-- core/
|   |-- constants/            # API sabitleri
|   |-- theme/                # Tema ve renkler
|   |-- utils/                # Router, yardimci fonksiyonlar
|-- models/
|   |-- book_model.dart       # Kitap modeli
|   |-- user_model.dart       # Kullanici modeli
|   |-- loan_model.dart       # Odunc modeli
|   |-- statistics_model.dart # Istatistik modeli
|   |-- *.freezed.dart        # Freezed generated
|   |-- *.g.dart              # JSON serializable
|-- services/
|   |-- api_client.dart       # Dio HTTP client
|   |-- auth_service.dart     # Kimlik dogrulama
|   |-- book_service.dart     # Kitap islemleri
|   |-- loan_service.dart     # Odunc islemleri
|   |-- statistics_service.dart # Istatistik API
|-- providers/
|   |-- auth_provider.dart    # Auth state
|   |-- book_provider.dart    # Book state
|   |-- loan_provider.dart    # Loan state
|   |-- statistics_provider.dart # Statistics state
|-- screens/
|   |-- auth/                 # Giris, Kayit, Splash
|   |-- user/                 # Ana sayfa, Profil, Kitap detay
|   |-- admin/                # Dashboard, Yonetim ekranlari
|-- widgets/                  # Yeniden kullanilabilir widgetlar
\end{lstlisting}

% ==================== STATE MANAGEMENT ====================
\section{State Management - Riverpod}

\subsection{Provider Yapisi}
Uygulama, Riverpod kutuphanesi ile state yonetimi yapmaktadir:

\begin{lstlisting}[language=Java]
// Ornek: Book Provider
final allBooksProvider = FutureProvider<List<BookModel>>((ref) async {
  final service = ref.watch(bookServiceProvider);
  return await service.getAllBooks();
});

// Ornek: Auth Provider
final authNotifierProvider = 
  StateNotifierProvider<AuthNotifier, AuthState>((ref) {
    return AuthNotifier(ref);
});
\end{lstlisting}

\subsection{Kullanilan Providerlar}
\begin{table}[H]
\centering
\begin{tabular}{|l|l|}
\hline
\textbf{Provider} & \textbf{Aciklama} \\
\hline
authNotifierProvider & Kullanici oturum durumu \\
currentUserProvider & Mevcut kullanici bilgisi \\
allBooksProvider & Tum kitaplar listesi \\
allLoansProvider & Tum odunc kayitlari \\
pendingLoansProvider & Bekleyen talepler \\
libraryStatisticsProvider & Kutuphane istatistikleri \\
categoryStatisticsProvider & Kategori istatistikleri \\
topBooksProvider & Populer kitaplar \\
\hline
\end{tabular}
\caption{Riverpod Provider Listesi}
\end{table}

% ==================== API ENTEGRASYONU ====================
\section{API Entegrasyonu}

\subsection{Dio HTTP Client}
API istekleri Dio kutuphanesi ile yapilmaktadir:

\begin{lstlisting}[language=Java]
class ApiClient {
  late final Dio _dio;
  final FlutterSecureStorage _secureStorage;
  
  ApiClient._internal() {
    _dio = Dio(BaseOptions(
      baseUrl: ApiConstants.baseUrl,
      connectTimeout: Duration(seconds: 30),
    ));
    
    // JWT Token interceptor
    _dio.interceptors.add(InterceptorsWrapper(
      onRequest: (options, handler) async {
        final token = await getToken();
        if (token != null) {
          options.headers['Authorization'] = 'Bearer $token';
        }
        return handler.next(options);
      },
    ));
  }
}
\end{lstlisting}

\subsection{API Servisleri}
\begin{table}[H]
\centering
\begin{tabular}{|l|l|}
\hline
\textbf{Servis} & \textbf{Islevler} \\
\hline
AuthService & login, register, getCurrentUser \\
BookService & getAllBooks, getBookById, search \\
LoanService & createLoan, returnBook, getMyLoans \\
CategoryService & getCategories, getActiveCategories \\
StatisticsService & getStatistics, getCategoryStats, getTopBooks \\
\hline
\end{tabular}
\caption{API Servisleri}
\end{table}

% ==================== EKRANLAR ====================
\section{Uygulama Ekranlari}

\subsection{Auth Ekranlari}
\begin{itemize}
    \item \textbf{SplashScreen:} Uygulama acilis, token kontrolu
    \item \textbf{LoginScreen:} Kullanici girisi (email/sifre)
    \item \textbf{RegisterScreen:} Yeni kullanici kaydi
\end{itemize}

\subsection{User Ekranlari}
\begin{itemize}
    \item \textbf{HomeScreen:} Kitap listesi, arama, kategori filtre
    \item \textbf{BookDetailScreen:} Kitap detayi, odunc alma butonu
    \item \textbf{ProfileScreen:} Profil bilgileri, odunc gecmisi
\end{itemize}

\subsection{Admin Ekranlari}
\begin{itemize}
    \item \textbf{AdminDashboardScreen:} Ozet istatistikler, hizli erisim
    \item \textbf{BookManagementScreen:} Kitap CRUD islemleri
    \item \textbf{LoanManagementScreen:} Odunc onay/red islemleri
    \item \textbf{StatisticsScreen:} Grafikli istatistik raporu
\end{itemize}

% ==================== GRAFIKLER ====================
\section{Istatistik Grafikleri}

\subsection{fl\_chart Entegrasyonu}
Uygulama, fl\_chart kutuphanesi ile grafik gosterimi yapmaktadir:

\begin{itemize}
    \item \textbf{PieChart:} Kategori bazli kitap dagilimi
    \item \textbf{LineChart:} Aylik odunc trendi
    \item \textbf{Stat Cards:} Ozet istatistik kartlari
\end{itemize}

\begin{lstlisting}[language=Java]
// Ornek: PieChart kullanimi
PieChart(
  PieChartData(
    sections: categories.map((cat) => 
      PieChartSectionData(
        value: cat.bookCount.toDouble(),
        title: cat.categoryName,
        color: colors[index],
      )
    ).toList(),
  ),
)
\end{lstlisting}

% ==================== FREEZED ====================
\section{Model Generation - Freezed}

\subsection{Immutable Data Classes}
Freezed kutuphanesi ile immutable ve type-safe modeller olusturulmaktadir:

\begin{lstlisting}[language=Java]
@freezed
class BookModel with _$BookModel {
  const factory BookModel({
    required int id,
    required String title,
    required String author,
    required int categoryId,
    required String categoryName,
    String? isbn,
    int? publishYear,
    String? imageUrl,
    @Default(true) bool isAvailable,
  }) = _BookModel;

  factory BookModel.fromJson(Map<String, dynamic> json) =>
      _$BookModelFromJson(json);
}
\end{lstlisting}

\subsection{Modeller}
\begin{itemize}
    \item BookModel - Kitap verisi
    \item UserModel - Kullanici verisi
    \item LoanModel - Odunc verisi
    \item CategoryModel - Kategori verisi
    \item LibraryStatistics - Genel istatistikler
    \item CategoryStatistics - Kategori istatistikleri
    \item TopBook - Populer kitaplar
\end{itemize}

% ==================== NAVIGATION ====================
\section{Navigation - go\_router}

\subsection{Route Yapisi}
\begin{lstlisting}[language=Java]
final routerProvider = Provider<GoRouter>((ref) {
  return GoRouter(
    initialLocation: '/splash',
    routes: [
      GoRoute(path: '/splash', builder: (_, __) => SplashScreen()),
      GoRoute(path: '/auth/login', builder: (_, __) => LoginScreen()),
      GoRoute(path: '/auth/register', builder: (_, __) => RegisterScreen()),
      GoRoute(path: '/home', builder: (_, __) => HomeScreen()),
      GoRoute(path: '/book/:id', builder: (_, state) => 
        BookDetailScreen(bookId: state.pathParameters['id']!)),
      GoRoute(path: '/admin', builder: (_, __) => AdminDashboardScreen()),
      GoRoute(path: '/admin/statistics', builder: (_, __) => 
        StatisticsScreen()),
    ],
  );
});
\end{lstlisting}

% ==================== SONUC ====================
\section{Sonuc ve Degerlendirme}

\subsection{Tamamlanan Ozellikler}
\begin{itemize}
    \item JWT tabanli kimlik dogrulama
    \item Kitap listeleme, arama, filtreleme
    \item Kitap odunc talebi olusturma ve takip
    \item Admin paneli ile tam yonetim
    \item Grafikli istatistik goruntulemem
    \item Responsive ve modern UI tasarimi
    \item Cross-platform destek (Android/iOS)
\end{itemize}

\subsection{Ogrenilen Konular}
\begin{itemize}
    \item Flutter widget yapisi ve lifecycle
    \item Riverpod ile state management
    \item Freezed ile code generation
    \item Dio ile REST API entegrasyonu
    \item go\_router ile declarative navigation
    \item fl\_chart ile veri gorsellestirme
\end{itemize}

% ==================== KAYNAKLAR ====================
\section{Kaynaklar}
\begin{enumerate}
    \item Flutter Documentation - \url{https://flutter.dev/docs}
    \item Riverpod Documentation - \url{https://riverpod.dev}
    \item Freezed Package - \url{https://pub.dev/packages/freezed}
    \item Dio Package - \url{https://pub.dev/packages/dio}
    \item fl\_chart Package - \url{https://pub.dev/packages/fl_chart}
    \item go\_router Package - \url{https://pub.dev/packages/go_router}
\end{enumerate}

\end{document}
